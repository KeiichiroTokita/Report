\documentclass[a4paper,10pt,twocolumn,fleqn]{jarticle}
%\usepackage{common_preamble.sty}
\usepackage{houkoku}

%%%%%%%%%%% 報告書番号 %%%%%%%%%%%%%
\newcommand{\Reportnum}[0]{1}%報告書番号

\makeatletter
\def\ps@myplain{%
 \let\@mkboth\@gobbletwo
 \def\@oddhead{ K.Tokita-Task-\thepage} %奇数ページヘッダ,名前の前のスペースは消さないように
 \def\@oddfoot{ \thepage \ }%奇数ページフッタ
 \let\@evenhead\@oddhead %偶数ページヘッダ
 \let\@evenfoot\@oddfoot %偶数ページフッタ\@topnewpage
}
\makeatother
\pagestyle{myplain} %ページスタイル
\graphicspath{{./fig/}}

%%%%%%%%%%%%%%%%%%%%%%%%%%%%%%%%%%%%%%%%%%%%%%%%%%%%%%%%%%%%%%%

\renewcommand{\thesubsection}{問\arabic{subsection}}
\renewcommand{\figurename}{Fig}
\renewcommand{\tablename}{Table}


\begin{document}
%\onecolumn
\twocolumn[
\begin{flushright}
\today\\
東京大学 堀藤本研究室 時田 圭一郎
\end{flushright}
\begin{center}
{\LARGE 環境構築勉強会 課題}
\end{center}
\vspace{2zw}
]

\section{序論}
  本稿では,環境構築勉強会の課題およびその途中経過と解を示す.本課題は課題1(MATLABとSimulinkによるシミュレーション)と課題2(C言語によるシミュレーション)からなるが,課題1についてはサンプルコードの実行が主であり,課題2の中でも課題1の内容が必要となるため,本稿で改めて記述することはしない.

\section{課題2(C言語によるシミュレーション)の解答}
  課題2では,簡単なRLC回路を流れる電流Iについて,以下の小問に従ってC言語でシミュレーションを行う.

  \subsection{RLC回路図を描け.}
    PowerPoint\textregistered を用いて,指定されたRLC回路図を描く.Fig\ref{RLC_circuit}に解答を示す.

    \begin{figure}[htbp]
      \begin{center}
        \fbox{
          \includegraphics[width = 50mm]{first_task_circuit-crop.pdf}
        }
        \caption{RLC circuit}
        \label{RLC_circuit}
      \end{center}
    \end{figure}

  \subsection{回路方程式と状態方程式を求めよ.}
    抵抗・コイル・キャパシタはそれぞれ$R, sL, \dfrac{1}{sC}$のインピーダンスを持つ.オームの法則より,回路方程式は(\ref{circuit_eq})式の通りとなる.
    \begin{equation}
      \label{circuit_eq}
      E = RI + sLI + \frac{1}{sC} I
    \end{equation}

    この回路方程式に対して状態空間表現を行う.状態変数として電流$I$を$X_1$,キャパシタにかかる電圧を$X_2$とすると,状態方程式は(\ref{state_eq})式の通りとなる.
    \begin{eqnarray}
      \label{state_eq}
        s
        \left[
          \begin{array}{c}
            X_1 \\
            X_2 \\
          \end{array}
        \right]
        =
        \left[
          \begin{array}{cc}
            -\frac{R}{L} & -\frac{1}{L} \\
            \frac{1}{C} & 0 \\
          \end{array}
        \right]
        \left[
          \begin{array}{c}
            X_1 \\
            X_2 \\
          \end{array}
        \right]
        +
        \left[
          \begin{array}{c}
            \frac{1}{L} \\
            0 \\
          \end{array}
        \right]
        E
    \end{eqnarray}

  \subsection{RLC回路についてRunge=Kutta法を用いたシミュレーションを行いなさい,その結果をグラフで示せ.}

  与えられたソースコードを使い,Runge=Kutta法を用いたシミュレーションを行った.用いた回路定数の値をTable\ref{values}に示す.シミュレーションの結果,(\ref{state_eq})式における状態変数$X_1$, $X_2$の時間変化はそれぞれFig\ref{RungeX1}, \ref{RungeX2}のようになった.電圧はキャパシタにかかる電圧は1Vへ,回路に流れる電流は0Aへと振動的に収束していることが確認できる.

  \begin{table}[htp]
    \centering
    \caption{The constant values of the circuit}
    \label{values}
    \begin{tabular}{|l|l|}
      \hline
      $E$ & 1.0[V] (t $>$ 0)\\ \hline
      $R$ & 1.0[$\Omega$]  \\ \hline
      $L$ & 3.0[mH] \\ \hline
      $C$ & 4.7[$\mu$ F]  \\ \hline
    \end{tabular}
  \end{table}


  \begin{figure}[htbp]
    \begin{center}
      \includegraphics[width = 60mm]{RungeX1.pdf}
      \caption{The change of the state variable $X_1$}
      \label{RungeX1}
    \end{center}
  \end{figure}

  \begin{figure}[htbp]
    \begin{center}
      \includegraphics[width = 60mm]{RungeX2.pdf}
      \caption{The change of the state variable $X_2$}
      \label{RungeX2}
    \end{center}
  \end{figure}




\end{document}
