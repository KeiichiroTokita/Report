%\documentclass[twocolumn,10pt]{jarticle} 
%\setlength{\columnsep}{3zw} 
\documentclass[]{jsarticle}
\usepackage[dvipdfmx]{graphicx}
\usepackage{amsmath,amssymb}

\title{後期実験総合レポート\\『課題番号22:核融合・宇宙プラズマ実験』}
\author{学籍番号03-170522 時田 圭一郎}
\date{実験日時:2017年9月26日 - 2017年10月5日\\
提出:2017年10月19日}

\begin{document}
\maketitle

\section{序論}
物質は温度の上昇に伴い固体, 液体, 気体と変化するが, さらに高温にすると第四態と言われるプラズマに変化する. 電子と原子核が互いを拘束することができず, 正負の電荷が混然一体となった状態で高速運動するプラズマは, 宇宙空間における物質の大部分を占めているほか, 現在研究が進められている核融合発電の基礎となっている. 『課題番号22:核融合・宇宙プラズマ実験』では5日間の様々な実験を通して, 核融合発電で行われている磁場によるプラズマの閉じ込めを中心に, 完全電離プラズマの特性やその計測法を理解することを目的とする. 本稿では, それらの実験の詳細やその結果から導かれる考察について述べる. 

\section{実験の原理および方法}
	\subsection{プラズマ閉じ込めの観察\label{day1} }
	
	\begin{figure}[htbp]
	\begin{center}
	\includegraphics[width=100mm]{souchi}
	\caption{プラズマ閉じ込め実験用装置\cite{t}}
	\label{souchi}
	\end{center}
	\end{figure}
	
	核融合を起こすためには, プラズマ状態の分子を局所的に集めることが必要である. そこで, 電離状態のプラズマが磁界により力を受けることを利用する. この実験では図\ref{souchi}の写真に示したプラズマ閉じ込め実験用真空容器を用いて, プラズマを磁界で閉じ込められるかを観察する. 装置の構成は以下の通りである\cite{t}. 
	
	\begin{enumerate}
		\item トロイダル磁界コイル(1コイル構成, 80ターン)
		\item 内部ポロイダルコイル(2コイル構成, 位置可変, 各々160ターン)
		\item 垂直磁界コイル(2コイル構成, 各々300ターン)
		\item グロー放電
		\item 直流電源(1-3用各1台, 10秒限時運転)
		\item 真空容器(円筒上下部ステンレス製, 円筒側面部アクリル製)
		\item 真空ポンプ(ロータリーポンプ, 油拡散ポンプ)$10^{-1} - 10^{-3} \rm{Torr}$	
	\end{enumerate}
	
	上述の通りこの装置には3種類のコイルが存在し, 異なる磁界を生成することが可能である. そのコイルとそれが作り出す磁界の概形を図\ref{gaikei}に示す. 
	
	\begin{figure}[htbp]
	\begin{center}
	\includegraphics[width=120mm]{gaikei}
	\caption{3種類のコイルの概形}
	\label{gaikei}
	\end{center}
	\end{figure}

	
	(a) 真空ポンプを用いて, 真空容器内の圧力を約0.3Torrまで下げる. グロー放電を点灯し, 目視により以下の5つの条件で, プラズマが磁力線によってどのように閉じ込められたかを観察する. 
	\begin{enumerate}
		\renewcommand{\labelenumi}{\Alph{enumi})}
		\item 外部垂直磁界コイルのみを励磁
		\item 中心トロイダルコイルのみを励磁
		\item 内部コイルのみを励磁
		\item 中心トロイダルコイル, 内部コイルを励磁
		\item 中心トロイダルコイル, 内部コイル, 外部垂直磁界コイルを励磁
	\end{enumerate}
	
	(b) 次に, 内部コイル電流を固定し, 外部垂直磁界コイルを正方向, 逆方向に励磁させた時の, プラズマ閉じ込め特性の変化を観察する. ただし, 正方向は内部コイルと外部垂直磁界コイルに流す電流の向きが逆向きである時の方向と定義する. 
	
	(c) 最後に, 外部垂直磁界コイル電流と内部コイル電流を固定し, 中心トロイダルコイル電流を変化させた時の, プラズマ閉じ込め特性の変化を観察する. 
	
	\subsection{静電プローブによるプラズマの計測\label{seiden}}
	
		\begin{figure}[htbp]
		\begin{minipage}{0.5\hsize}
			\begin{center}
				\includegraphics[width=60mm]{probe}
				\caption{ダブル静電プローブの回路\cite{t}}
				\label{probe}
			\end{center}
		\end{minipage}
		\begin{minipage}{0.5\hsize}
			\begin{center}
				\includegraphics[width=60mm]{gaikei}
				\caption{ダブル静電プローブのV-I特性\cite{t}}
				\label{gaikei}
			\end{center}
		\end{minipage}
	\end{figure}

	
	
	図\ref{probe}に示すように, 2本の探針をプラズマ中に挿入し, 直流電圧$V$を印加して, 電圧-電流(V-I)特性を計測することにより, プラズマの電子密度$n_E$と電子温度$T_e$を計測することができる. V-I特性は印加電圧$V$に対して電流$I$はほぼ線形に増加し, その特性は$V=0$に対して対称となる. グラフの概形を図\ref{gaikei}に示す. 電圧が$V_S$以上の場合は電流が飽和し(イオン飽和電流$I_{is}$と呼ばれる), その大きさは式\ref{I_is}のように表される. ここで$S$は探針の側面積である. 
	\begin{equation}
		\label{I_is}
		I_{is} \simeq 0.6en_e S{(kT_e / m_i)}^{1/2} 
	\end{equation}
	式\ref{I_is}より電子密度$n_e$が求められる.  V-I特性のグラフの原点における傾き$\alpha$について, 
	\begin{equation}
		\label{T_e}
		\alpha = (e / kT_e)I_{is}
	\end{equation}
	という式が成り立つ.  この式を用いると, 電子温度$T_e$が求められる\cite{t}. \\
	この実験では探針直径1mm, 長さ9mmの静電プローブを用いて, プラズマ電流の電圧電流特性(V-I特性)を求め, そこから電子温度$T_e$, 電子密度$n_e$を求める. 測定する電圧の範囲は$-15[\rm{V}] 〜 15[\rm{V}]$とする. 磁場を加えなかった場合と, \ref{day1}節で述べた5パターンの磁場を加えた場合の計6通りについて, それらの値がどのように異なるかを観測し, どの条件のプラズマの電子密度が高いのかを実験により検証する. 
	
	\subsection{磁気プローブによる磁場計測\label{jiki}}
	軸(z)方向, トロイダル(t)方向, 半径(r)方向を向けたピックアップコイル(外径5mm, 300ターン)をガラス管内に設置し, オシロスコープを用いて電圧を計測する. 
	面積$S$, 巻数$N$のピックアップコイルを, 時間変化する磁界$B$に挿入すると, ファラデーの電磁誘導の法則より電圧$V(t) = SN(dB(t) / dt)$が得られるから, $dB(t) / dt = V(t) / SN$を時間で数値積分することにより磁場$B(t)$が計測できる. 図\ref{gaikei}に見られるように, z, r, t方向を向けたピックアップコイルではそれぞれEF, TF, PFの磁場を計測できることが予想される. 
	
	今回の実験では, 一方向のピックアップコイルについて\ref{day1}節で述べた5つの条件で磁場を印加し, 計測した電圧から上記の方法で磁場を計測する. またz軸方向の磁界分布を調べるため, プローブをガラス管の最も奥まで差し込んだ時を$z = 0[\rm{cm}]$とし, そこから鉛直上方に$z = 2, 4[\rm{cm}]$引き上げた計3点について計測する. すなわち$3(ピックアップコイルの方向)\times 5(磁場の印加の仕方)\times 3(ピックアップコイルの高さ)= 45$通りの計測を行う. 

	\subsection{分光計測}
	光ファイバーを用いた分光計測を行う. 発行強度が大きくなる磁場の条件でプラズマの発光を計測し, 発光スペクトルを求める. ここから, プラズマの発光が何の原子によるものなのかを推定する. 

\section{結果\label{results}}
	\subsection{プラズマ閉じ込めの観察\label{day1_results}}
	(a) \ref{day1}節で述べた通り, 目視により以下の5つの条件で, プラズマが磁力線によってどのように閉じ込められたかを観察した. 観察されたプラズマの写真を図\ref{day1_A}-\ref{day1_E}に示す. 一部の磁場のみしか励磁しなかったときはプラズマが外側に発散してしまうのに対し, E)で全ての磁場を励磁したときはプラズマが中心に集まっており, 磁場により閉じ込められているのがわかる. 
	\begin{enumerate}
		\renewcommand{\labelenumi}{\Alph{enumi})}
		\item 外部垂直磁界コイルのみを励磁
		\item 中心トロイダルコイルのみを励磁
		\item 内部コイルのみを励磁
		\item 中心トロイダルコイル, 内部コイルを励磁
		\item 中心トロイダルコイル, 内部コイル, 外部垂直磁界コイルを励磁
	\end{enumerate}

	\begin{figure}[htbp]
		\begin{minipage}{0.33\hsize}
			\begin{center}
			\fbox{
				\includegraphics[width=50mm, angle=-90]{day1_A}
				}
				\caption{A)外部垂直磁界コイルのみを励磁}
				\label{day1_A}
			\end{center}
		\end{minipage}
		\begin{minipage}{0.33\hsize}
			\begin{center}
			\fbox{
				\includegraphics[width=50mm, angle=-90]{day1_B}
				}
				\caption{B)中心トロイダルコイルのみを励磁}
				\label{day1_B}
			\end{center}
		\end{minipage}
		\begin{minipage}{0.33\hsize}
			\begin{center}
			\fbox{
				\includegraphics[width=50mm, angle=-90]{day1_C}
				}
				\caption{C)内部コイルのみを励磁}
				\label{day1_C}
			\end{center}
		\end{minipage}
		
	\end{figure}
	
	\begin{figure}[htbp]
		\begin{minipage}{0.33\hsize}
			\begin{center}
			\fbox{
				\includegraphics[width=50mm, angle=-90]{day1_D}
				}
				\caption{D)中心トロイダルコイル, 内部コイルを励磁}
				\label{day1_D}
			\end{center}
		\end{minipage}
		\begin{minipage}{0.33\hsize}
			\begin{center}
			\fbox{
				\includegraphics[width=50mm, angle=-90]{day1_E}
				}
				\caption{E)中心トロイダルコイル, 内部コイル, 外部垂直磁界コイルを励磁}
				\label{day1_E}
			\end{center}
		\end{minipage}
	\end{figure}
	
	(b) 内部コイル電流を固定し, 外部垂直磁界コイルを正方向, 逆方向に励磁させた時の, プラズマ閉じ込め特性の変化を観察した. 観察されたプラズマの写真を図\ref{day1_sei}, \ref{day1_gyaku}に示す. 正方向の時のみプラズマが閉じ込められていることがわかる. 
	
	\begin{figure}[htbp]
		\begin{minipage}{0.5\hsize}
			\begin{center}
			\fbox{
				\includegraphics[width=50mm, angle=-90]{day1_sei}
				}
				\caption{A)EFを正方向に励磁}
				\label{day1_sei}
			\end{center}
		\end{minipage}
		\begin{minipage}{0.5\hsize}
			\begin{center}
			\fbox{
				\includegraphics[width=50mm, angle=-90]{day1_gyaku}
				}
				\caption{B)EFを逆方向に励磁}
				\label{day1_gyaku}
			\end{center}
		\end{minipage}
	\end{figure}

	(c)最後に, 外部垂直磁界コイル電流と内部コイル電流を固定し, 中心トロイダルコイル電流を変化させた時の, プラズマ閉じ込め特性の変化を観察する. 観察されたプラズマの写真を図\ref{variation}に示す. プラズマの形状にはあまり変化が見られなかった. 
	
	\begin{figure}[htbp]
	\fbox{
		\begin{minipage}{0.33\hsize}
			\begin{center}
				\includegraphics[width=50mm, angle=-90]{day1_C1}
			\end{center}
		\end{minipage}
		\begin{minipage}{0.33\hsize}
			\begin{center}
				\includegraphics[width=50mm, angle=-90]{day1_C2}
			\end{center}
		\end{minipage}
		\begin{minipage}{0.33\hsize}
			\begin{center}
				\includegraphics[width=50mm, angle=-90]{day1_C3}
			\end{center}
		\end{minipage}
		}
		\caption{左から徐々に中心トロイダルコイル電流を減少させた時のプラズマの変化}
		\label{variation}
	\end{figure}

	\subsection{静電プローブによるプラズマの計測\label{seiden_results}}
	\ref{seiden}節で述べた通り, プラズマ電流の電圧電流特性(V-I特性)を求め, そこから電子温度$T_e$, 電子密度$n_e$を求める. 磁場を加えなかった場合と, \ref{day1}節で述べた5パターンの磁場を加えた場合の計6通りについて, それらの値がどのように異なるかを観測し, どの条件のプラズマの電子密度が高いのかを実験により検証する.
	(1)磁場を加えなかった場合
	
		電圧電流特性のグラフでの原点における傾きを考えるため, 特性は$V = 0$で連続である必要がある. しかし電圧が0Vの時も, 繋ぐ方向によって異なるオフセット($10^{-7}[\rm{A}]$のオーダー)が生じるため,  計測された値をそのまま使うと連続にならない. そこで, オフセットの差分を電流値とすることにより, 電圧が0Vの時に電流が0Aとなるよう調節した値を用いて解析を行う(他の条件における測定でも同様). 図\ref{nomag}にその結果のグラフを示す. 	
	\begin{figure}[htbp]
	\begin{center}
	\includegraphics[width=100mm]{nomag}
	\caption{磁場を加えなかった場合の電圧電流特性}
	\label{nomag}
	\end{center}
	\end{figure}
	
	グラフより, 原点における接線の傾きは$\alpha \simeq 1.6\times10^{-7}[/\rm{\Omega}]$, イオン飽和電流は$I_{is} \simeq 2.5\times10^{-7}[\rm{A}]$と求められる. 式(\ref{T_e})において, 電気素量$e = 1.6 \times 10^{-19}[\rm{C}]$, ボルツマン定数$k = 1.38 \times 10^{-23}[\rm{J/K}]$として計算すると,電子温度は$T_e = 1.81 \times 10^4[\rm{K}]$となる. 同様に式(\ref{I_is})においてイオンの質量を$m_i = 14.4 / (6.02 \times 10^{23} [\rm{g}] \simeq 2.39 \times 10^{-26}[\rm{kg}]$とし(気体は窒素分子80\%, 酸素分子20\%からなると近似), プローブの断面積を$S = 2\pi \times 0.5[\rm{mm}] \times 9[mm] \simeq 2.83 \times 10^{-7} [\rm{m^2}]$として計算すると, 電子密度は$n_e \simeq 2.85 \times 10^{15}[\rm{/m^3}]$と求められる. 
	
	(2)一部の磁場のみ加えた場合
	
		\ref{day1}節で述べた磁場のうち, A)からD), すなわち一部の磁場のみを加えたときは, 電流の値のオーダーが全ての磁場を加えたE)に比べて1桁少なく, ダブル静電プローブの特性が現れなかった. 図\ref{seidenA}-\ref{seidenD}にその結果のグラフを示す. 原点における接線の傾き$\alpha$, イオン飽和電流$I_{is}$をグラフから求めることができないため, 式(\ref{I_is})(\ref{T_e})を用いて電子温度, 電子密度を求めることができないが, 電流のオーダーから考えて非常に小さい値であると考えられる.
		
		\begin{figure}[htbp]
		\begin{minipage}{0.5\hsize}
			\begin{center}
				\includegraphics[width=60mm]{seidenA}
				\caption{A) EFのみを加えたときのV-I特性}
				\label{seidenA}
			\end{center}
		\end{minipage}
		\begin{minipage}{0.5\hsize}
			\begin{center}
				\includegraphics[width=60mm]{seidenB}
				\caption{B) TFのみを加えたときのV-I特性}
				\label{seidenB}
			\end{center}
		\end{minipage}
	\end{figure}
	
	\begin{figure}[htbp]
		\begin{minipage}{0.5\hsize}
			\begin{center}
				\includegraphics[width=60mm]{seidenC}
				\caption{C) PFのみを加えたときのV-I特性}
				\label{seidenC}
			\end{center}
		\end{minipage}
		\begin{minipage}{0.5\hsize}
			\begin{center}
				\includegraphics[width=60mm]{seidenD}
				\caption{D) TF, PFを加えたときのV-I特性}
				\label{seidenD}
			\end{center}
		\end{minipage}
	\end{figure}
		
		 
		
	(3)全ての磁場を加えた場合
	
		\ref{day1}節で述べた磁場のパターンのうち, 3つ全ての磁場を加えたときは, ダブル静電プローブの特性がよく確認された. 図\ref{seidenE}にその結果のグラフを示す. グラフより, 原点における接線の傾きは$\alpha \simeq 8.0\times10^{-7}[/\rm{\Omega}]$, イオン飽和電流は$I_{is} \simeq 1.2\times10^{-6}[\rm{A}]$と求められる. 
		\footnote{電圧を-15\rm{V}から順に0に近づけ, その後15\rm{V}へと上げる用にして計測を進めたが, +5\rm{V}を測る前に一度コイルの電源を切ったために電流の値がその前後で急激に変動した. そのため, イオン飽和電流の値は電流値が安定している, 電圧が負である点を用いて算出している. }. 
		これらの値より式(\ref{I_is})(\ref{T_e})を用いて計算すると, $T_e = 1.74 \times 10^4[\rm{K}]$, 電子密度は$n_e \simeq 1.39 \times 10^{16}[\rm{/m^3}]$と求められる. 
		
	\begin{figure}[htbp]
	\begin{center}
	\includegraphics[width=120mm]{seidenE}
	\caption{E) EF, TF, PFを全て加えたときのV-I特性}
	\label{seidenE}
	\end{center}
	\end{figure}

	
	\subsection{磁気プローブによる磁場計測\label{jiki_results}}
		\ref{jiki}節で述べた通り, 変えるパラメータとしてはピックアップコイルの方向, 磁場の印加の仕方, ピックアップコイルの高さの三つがある. 以下で示すのは電圧の波形であるが,  $NS$の値が$NS = 300 \times \pi \times (0.25[\rm{mm}])^2 $で一定なので, 磁場の波形は電圧の波形と同じ形になる. 
		
		\subsubsection{トロイダル(t)方向のコイルによる計測}
		はじめにコイルの方向をt方向, 位置を$z = 0$に固定して, 5パターンの磁場を加えた. 図\ref{EF_0_t}-\ref{TFPFEF_0_t}にそのグラフを示す. TFの波形は滑らかで連続的な変化なのに対し, EFやPFのみを印加した時の結果は変動が激しい上に値非常にが小さく, 磁場変化を正しく計測できていないのが確認できる. よって\ref{jiki}節で予想した通り, コイルをt方向に向けるとTFの磁場を計測できることがわかる. また複数の磁場を加えた時には, およそ波形がそれらの足し合わせの形になっていることが確認できる. 
		
		\begin{figure}[htbp]
	
		\begin{minipage}{0.33\hsize}
			\begin{center}
				\includegraphics[width=50mm, height=40mm]{EF_0_t}
				\caption{t方向, $z = 0$でのEF計測波形}
				\label{EF_0_t}
			\end{center}
		\end{minipage}
		\begin{minipage}{0.33\hsize}
			\begin{center}
				\includegraphics[width=50mm, height=40mm]{TF_0_t}
				\caption{t方向, $z = 0$でのTF計測波形}
				\label{TF_0_t}
			\end{center}
		\end{minipage}
		\begin{minipage}{0.33\hsize}
			\begin{center}
				\includegraphics[width=50mm, height=40mm]{PF_0_t}
				\caption{t方向, $z = 0$でのPF計測波形}
				\label{PF_0_t}
			\end{center}
		\end{minipage}
		
		\end{figure}
		
		\begin{figure}[htbp]
		\begin{minipage}{0.5\hsize}
			\begin{center}
				\includegraphics[width=50mm, height=40mm]{TFPF_0_t}
				\caption{t方向, $z = 0$でのTF, PF計測波形}
				\label{TFPF_0_t}
			\end{center}
		\end{minipage}
		\begin{minipage}{0.5\hsize}
			\begin{center}
				\includegraphics[width=50mm, height=40mm]{TFPFEF_0_t}
				\caption{t方向, $z = 0$でのTF, PF, EF計測波形}
				\label{TFPFEF_0_t}
			\end{center}
		\end{minipage}
	\end{figure}

次にt方向のコイルでTFのみを計測した波形を, 軸の高さ$z$を変えて比較する. そのグラフを図\ref{TF_0_t^}-\ref{TF_4_t}に示す. どの高さでもTFが計測できていることがわかる. 

	\begin{figure}[htbp]
	
		\begin{minipage}{0.33\hsize}
			\begin{center}
				\includegraphics[width=50mm, height=40mm]{TF_0_t}
				\caption{t方向, $z = 0$でのTF計測波形}
				\label{TF_0_t^}
			\end{center}
		\end{minipage}
		\begin{minipage}{0.33\hsize}
			\begin{center}
				\includegraphics[width=50mm, height=40mm]{TF_2_t}
				\caption{t方向, $z = 2$でのTF計測波形}
				\label{TF_2_t}
			\end{center}
		\end{minipage}
		\begin{minipage}{0.33\hsize}
			\begin{center}
				\includegraphics[width=50mm, height=40mm]{TF_4_t}
				\caption{t方向, $z = 4$でのTF計測波形}
				\label{TF_4_t}
			\end{center}
		\end{minipage}
		
	\end{figure}
		
		\subsubsection{半径(r)方向のコイルによる計測}
		
		はじめにコイルの方向をr方向, 位置を$z = 0$に固定して, 5パターンの磁場を加えた. 図\ref{EF_0_r}-\ref{TFPFEF_0_r}にそのグラフを示す. PFの波形は滑らかで連続的な変化なのに対し, EFやTFのみを印加した時の結果は変動が激しい上に値が非常に小さく, 磁場変化を正しく計測できていないのが確認できる. よって\ref{jiki}節で予想した通り, コイルをr方向に向けるとPFの磁場を計測できることがわかる. また複数の磁場を加えた時には, t方向の時と同様にしておよそ波形がそれらの足し合わせの形になっていることが確認できる. 
		
		\begin{figure}[htbp]
	
		\begin{minipage}{0.33\hsize}
			\begin{center}
				\includegraphics[width=50mm, height=40mm]{EF_0_r}
				\caption{r方向, $z = 0$でのEF計測波形}
				\label{EF_0_r}
			\end{center}
		\end{minipage}
		\begin{minipage}{0.33\hsize}
			\begin{center}
				\includegraphics[width=50mm, height=40mm]{TF_0_r}
				\caption{r方向, $z = 0$でのTF計測波形}
				\label{TF_0_r}
			\end{center}
		\end{minipage}
		\begin{minipage}{0.33\hsize}
			\begin{center}
				\includegraphics[width=50mm, height=40mm]{PF_0_r}
				\caption{r方向, $z = 0$でのPF計測波形}
				\label{PF_0_r}
			\end{center}
		\end{minipage}
		
		\end{figure}
		
		\begin{figure}[htbp]
		\begin{minipage}{0.5\hsize}
			\begin{center}
				\includegraphics[width=50mm, height=40mm]{TFPF_0_r}
				\caption{r方向, $z = 0$でのTF, PF計測波形}
				\label{TFPF_0_r}
			\end{center}
		\end{minipage}
		\begin{minipage}{0.5\hsize}
			\begin{center}
				\includegraphics[width=50mm, height=40mm]{TFPFEF_0_r}
				\caption{r方向, $z = 0$でのTF, PF, EF計測波形}
				\label{TFPFEF_0_r}
			\end{center}
		\end{minipage}
	\end{figure}

次にr方向のコイルでTFのみを計測した波形を, 軸の高さ$z$を変えて比較する. そのグラフを図\ref{TF_0_r^}-\ref{TF_4_r}に示す. どの高さでもPFが計測できていることがわかる. 

	\begin{figure}[htbp]
	
		\begin{minipage}{0.33\hsize}
			\begin{center}
				\includegraphics[width=50mm, height=40mm]{PF_0_r}
				\caption{r方向, $z = 0$でのPF計測波形}
				\label{PF_0_r^}
			\end{center}
		\end{minipage}
		\begin{minipage}{0.33\hsize}
			\begin{center}
				\includegraphics[width=50mm, height=40mm]{PF_2_r}
				\caption{r方向, $z = 2$でのPF計測波形}
				\label{PF_2_r}
			\end{center}
		\end{minipage}
		\begin{minipage}{0.33\hsize}
			\begin{center}
				\includegraphics[width=50mm, height=40mm]{PF_4_r}
				\caption{r方向, $z = 4$でのPF計測波形}
				\label{PF_4_r}
			\end{center}
		\end{minipage}
		
	\end{figure}
	
		
		\subsubsection{軸(z)方向のコイルによる計測}
		
		はじめにコイルの方向をz方向, 位置を$z = 0$に固定して, 5パターンの磁場を加えた. 図\ref{EF_0_z}-\ref{TFPFEF_0_z}にそのグラフを示す. EFの波形に比べ, TFやPFのみを印加した時の結果は変動が激しい上に値が非常に小さく, 磁場変化を正しく計測できていないのが確認できる. よって\ref{jiki}節で予想した通り, コイルをz方向に向けるとEFの磁場を計測できることがわかる. また複数の磁場を加えた時には, t, r方向の時と同様にしておよそ波形がそれらの足し合わせの形になっていることが確認できる. 
		
		\begin{figure}[htbp]
	
		\begin{minipage}{0.33\hsize}
			\begin{center}
				\includegraphics[width=50mm, height=40mm]{EF_0_z}
				\caption{z方向, $z = 0$でのEF計測波形}
				\label{EF_0_z}
			\end{center}
		\end{minipage}
		\begin{minipage}{0.33\hsize}
			\begin{center}
				\includegraphics[width=50mm, height=40mm]{TF_0_z}
				\caption{z方向, $z = 0$でのTF計測波形}
				\label{TF_0_z}
			\end{center}
		\end{minipage}
		\begin{minipage}{0.33\hsize}
			\begin{center}
				\includegraphics[width=50mm, height=40mm]{PF_0_z}
				\caption{z方向, $z = 0$でのPF計測波形}
				\label{PF_0_z}
			\end{center}
		\end{minipage}
		
		\end{figure}
		
		\begin{figure}[htbp]
		\begin{minipage}{0.5\hsize}
			\begin{center}
				\includegraphics[width=50mm, height=40mm]{TFPF_0_z}
				\caption{z方向, $z = 0$でのTF, PF計測波形}
				\label{TFPF_0_z}
			\end{center}
		\end{minipage}
		\begin{minipage}{0.5\hsize}
			\begin{center}
				\includegraphics[width=50mm, height=40mm]{TFPFEF_0_z}
				\caption{z方向, $z = 0$でのTF, PF, EF計測波形}
				\label{TFPFEF_0_z}
			\end{center}
		\end{minipage}
	\end{figure}

次にz方向のコイルでEFのみを計測した波形を, 軸の高さ$z$を変えて比較する. そのグラフを図\ref{EF_0_z^}-\ref{EF_4_z}に示す. 電圧の値よりどの高さでもEFが計測できていることがわかるが, $z = 0$での波形のみ値が収束していないことがわかる. 原因については\ref{consideration}節で考察する. 

	\begin{figure}[htbp]
	
		\begin{minipage}{0.33\hsize}
			\begin{center}
				\includegraphics[width=50mm, height=40mm]{EF_0_z}
				\caption{z方向, $z = 0$でのEF計測波形}
				\label{EF_0_z^}
			\end{center}
		\end{minipage}
		\begin{minipage}{0.33\hsize}
			\begin{center}
				\includegraphics[width=50mm, height=40mm]{EF_2_z}
				\caption{z方向, $z = 2$でのTF計測波形}
				\label{EF_2_z}
			\end{center}
		\end{minipage}
		\begin{minipage}{0.33\hsize}
			\begin{center}
				\includegraphics[width=50mm, height=40mm]{EF_4_z}
				\caption{z方向, $z = 4$でのTF計測波形}
				\label{EF_4_z}
			\end{center}
		\end{minipage}
		
	\end{figure}
		
	\subsection{分光計測\label{bunko_results}}
	プラズマの発光強度が大きくなるような磁場の条件で, 光ファイバーを用いた分光計測を行う. TF, PF, EFを全てかけた条件で, 図\ref{spectrum}で示すようなスペクトルが得られた. 390, 426, 468, 485, 655nmのところにラインが見られる. 
	
	\begin{figure}[htbp]
	\begin{center}
	\includegraphics[width=120mm]{spectrum}
	\caption{プラズマの発光スペクトル}
	\label{spectrum}
	\end{center}
	\end{figure}
		
	
\section{考察\label{consideration}}
	本節では\ref{results}節に示された結果に関して考察を行うとともに, \ref{tasks}節では実験テキスト(\cite{t})P490に示されている検討課題の解答を示す. 
	\subsection{プラズマ閉じ込めの観察}
		\ref{day1_results}節で示した結果より, プラズマを閉じ込めるためにはTF, PFにより螺旋状のトーラスを作った上で, EFによってトーラスが外側に広がるのを抑えることが必要であることがわかる. (c)の実験においては中心トロイダルコイルの電流を変化させた時のプラズマの形状を観察したが, あまり変化が見られなかった. これは電流の変化が不十分であった, あるいは他の磁場の影響が大きくTFの変化が見えなかったなどの原因が考えられる. 

	\subsection{静電プローブによるプラズマの計測}
	\ref{seiden_results}に示した結果より, TF, PF, EFを全て印加したときの電子密度が最も高くなることが確認できるが, 電子温度の変化については確認できなかった. 本実験では全てのパターンの磁場で電子温度, 密度を算出することが想定されていたが, 前述の通り不可能であった. 磁場を全てかけないとプラズマは閉じ込められず, 静電プローブで計測される場所にはほとんど電子が存在しないため, 静電プローブの理想的な特性が出ないものと考えられる. また出力の電圧が非常に小さく($10^{-7}$Vのオーダー), 計測誤差は非常に大きいと考えられる. 	
	
	\subsection{磁気プローブによる磁場計測}
	\ref{jiki_results}に示した結果より, z, r, t方向を向けたピックアップコイルではそれぞれEF, TF, PFの磁場を計測できることが確認できた. r, t方向を向けたピックアップコイルではz軸方向の磁界分布についての特徴が得られなかった一方, z方向を向けたピックアップコイルでは$z = 0$のときのみEFの波形が一定値で安定せず, 軸方向分布に特徴が見られた. これはコイルからの距離による結果だとも考えられるが, EFを出力する実験装置の調子が悪く, そのために同じ条件で計測が出来なかった結果である可能性も十分に考えられる. 
	
	\subsection{分光計測}
	\ref{bunko_results}に示した結果より, プラズマの発光スペクトルが何の分子によるものなのかを推定する. NIST Atomic Spectra Database Lines Form(\cite{NIST})によると485nm, 655nm付近のスペクトルはともに窒素分子によるものだと推測される. しかし, 窒素の次に空気中を占める割合が大きい酸素分子のスペクトルは確認できなかった. そのため, プラズマの発光においていくつか説明できないラインが存在する. プラズマの計測の際はアクリル板を挟んで行われることが一つの原因として考えられる. 

	
	\subsection{検討課題について\label{tasks}}
	以下の4問の検討課題について解答する. 
	
	\begin{enumerate}
	\item \ref{day1}節の(a)で示したプラズマ閉じ込めの観察について, 装置の大円周方向(トロイダル方向)の磁界だけでは, イオンも電子も閉じ込められないことを示せ. 
	\item プラズマの観察, 磁気プローブによる計測結果から\ref{day1}節で示したA)〜E)各々の配位の磁界配位の概形を描け. 
	\item \ref{day1}節の(b)の結果について各々の配位の磁界配位の概形を描け. 特にセパラトリクスと呼ばれる内部磁界と外部磁界をつなぐ磁力線の境界に注目せよ. また(c)について, トロイダルコイル電流を変化させるとプラズマの振る舞いにどのような影響があると考えられるか. 
	\item 単針型の静電プローブのV-I特性を表す理論式
	\begin{eqnarray}
		\label{vitheory}
		I(V) = \frac{1}{4} en_e S[v_i - v_e \rm{exp}\{\it{\alpha(V - V_s)}\}] 
	\end{eqnarray}
	より, ダブル静電プローブの特性を導け. 
	\end{enumerate}
		\subsubsection{課題1:トロイダル方向の磁界だけでは荷電粒子を閉じ込められない理由}
		トロイダル方向の磁界だけをかけたとき, 磁力線は図????%
		に示すような単純トーラスを作る. この磁界によって荷電粒子がどのような運動をするか考えたい. 
		$+q[\rm{C}]$の正電荷は, 速度$\textbf{v}$で運動するとき磁束$\textbf{B}$によって$q(\textbf{v} \times \textbf{B})$のローレンツ力を受ける. 荷電粒子の質量を$m$とすると, 遠心力$m\frac{|\textbf{v}|^2}{r}$がローレンツ力と釣り合うような半径$r$の等速円運動をする. すなわち, 
		\begin{eqnarray*}
			qvB = m\frac{v^2}{r}\\
			r = \frac{mv}{qB}
		\end{eqnarray*}
		ここで, $v$は円運動の速さ, $B$は磁束の大きさである. すなわち, 回転半径は磁束の大きさに反比例する. 
		さて, トロイダル磁場を作るためにコイルに電流$I$を流したとき, コイルから$l$だけ離れた点での磁束の大きさは$$B = \frac{\mu_0 I}{2\pi l}$$であり, $l$に反比例する. したがって, コイルから離れるほど磁界の大きさが小さく, 回転半径は大きい. したがって, 電荷は同一円周上を周るのではなく, 図\ref{gradB}に示すような挙動を示し, 正電荷と負電荷が上下に分離する. これを$\nabla \textbf{B}$ドリフトという. 
		$\nabla \textbf{B}$ドリフトにより電荷が上下に分離すると, その間に電界が生じる. 円運動をしている粒子に, この電界による力が加わるため, 運動は図\ref{EcrossB}に示すようなものになる. これを$\textbf{E} \times \textbf{B}$ドリフトという. 
		
		\begin{figure}[htbp]
		\begin{minipage}{0.5\hsize}
			\begin{center}
			\fbox{
				\includegraphics[width=60mm]{gradB}
				}
				\caption{$\nabla \textbf{B}$ドリフト}
				\label{gradB}
			\end{center}
		\end{minipage}
		\begin{minipage}{0.5\hsize}
			\begin{center}
			\fbox{
				\includegraphics[width=60mm]{EcrossB}
				}
				\caption{$\textbf{E} \times \textbf{B}$ドリフト}
				\label{EcrossB}
			\end{center}
		\end{minipage}
	        \end{figure}

				
		以上の2種類のドリフトにより, 荷電粒子はコイルから離れ続ける. そのため, トロイダル磁場のみによる単純トーラスでは荷電粒子を閉じ込められないことがわかる. 
		
		\subsubsection{課題2:五つの磁場パターンに対する磁界配位の概形について}
		\ref{day1}節で述べた五種類の磁界配位の概形を図\ref{EFimage}-\ref{Allimage}に示す. 
		
		\begin{figure}[htbp]
		\begin{minipage}{0.33\hsize}
			\begin{center}
			\fbox{
				\includegraphics[width=40mm]{EFimage}
				}
				\caption{A)外部垂直磁界コイルのみを励磁}
				\label{EFimage}
			\end{center}
		\end{minipage}
		\begin{minipage}{0.33\hsize}
			\begin{center}
			\fbox{
				\includegraphics[width=40mm]{TFimage}
				}
				\caption{B)中心トロイダルコイルのみを励磁}
				\label{TFimage}
			\end{center}
		\end{minipage}
		\begin{minipage}{0.33\hsize}
			\begin{center}
			\fbox{
				\includegraphics[width=40mm]{PFimage}
				}
				\caption{C)内部コイルのみを励磁}
				\label{PFimage}
			\end{center}
		\end{minipage}
		
	\end{figure}
	
	\begin{figure}[htbp]
		\begin{minipage}{0.33\hsize}
			\begin{center}
			\fbox{
				\includegraphics[width=40mm]{TFPFimage}
				}
				\caption{D)中心トロイダルコイル, 内部コイルを励磁}
				\label{TFPF_image}
			\end{center}
		\end{minipage}
		\begin{minipage}{0.33\hsize}
			\begin{center}
			\fbox{
				\includegraphics[width=40mm]{Allimage}
				}
				\caption{E)中心トロイダルコイル, 内部コイル, 外部垂直磁界コイルを励磁}
				\label{Allimage}
			\end{center}
		\end{minipage}
	\end{figure}

		
		\subsubsection{課題3:EF, TFの変化に対する磁界配位の変化について}
		\ref{day1}節の実験(b)では, EFのかける方向によってプラズマが閉じ込められるか否かが変わった. 図\ref{seiimage}, \ref{gyakuimage}にその概形を示す. 正方向にEFをかけたときはEFとPFが同じ向きであり, お互いの磁力線が独立しているのに対し, 逆方向にかけたときはEFとPFが逆向きで磁力線が閉じてしまう. そのために, 逆方向では磁力線でプラズマを閉じ込めておくことができないと考えられる. 
		
		\begin{figure}[htbp]
		\begin{minipage}{0.5\hsize}
			\begin{center}
			\fbox{
				\includegraphics[width=60mm]{seiimage}
				}
				\caption{EFを正方向にかけた時の磁界配位}
				\label{seiimage}
			\end{center}
		\end{minipage}
		\begin{minipage}{0.5\hsize}
			\begin{center}
			\fbox{
				\includegraphics[width=60mm]{gyakuimage}
				}
				\caption{EFを逆方向にかけた時の磁界配位}
				\label{gyakuimage}
			\end{center}
		\end{minipage}
	        \end{figure}

		
		\subsubsection{課題4:ダブル静電プローブの特性の導出}
		単針型の静電プローブのV-I特性を表す理論式(\ref{vitheory})より, ダブル静電プローブの特性を導く. 
		式(\ref{vitheory})の[ ]内が0になる$V$値を$V_f$とすると, この値で$I$値は0になる. この$V_f$を浮遊電位と呼ぶ. すなわち
			$$v_i - v_e \rm{exp} \{\it{\alpha(V_f - V_s)}\} \rm{= 0} $$
			\begin{eqnarray}
				\therefore
				\label{V_f}
				V_f = V_s + \frac{1}{\alpha} \rm{log} \it{ \left| \frac{v_i}{v_e} \right| }
			\end{eqnarray}
	である\cite{plasma}. 
		ダブルプローブの場合はプローブ系全体が電位的に浮いているので, 両探針間の印加電圧が0なら両探針ともに上記の浮遊電位にあり, 電流は流れない. 印加電圧が$V \neq 0$なら, 片方が浮遊電位の上, 他方が下に移動してイオン電流が流れるようになる. このとき電圧は$V_f + \frac{V}{2}, V_f - \frac{V}{2}$であり, $V > 0$なら$I\left(V_f - \frac{V}{2}\right)$の電流が流れる. 式(\ref{vitheory}), (\ref{V_f})より, 
			\begin{eqnarray*}
				I\left(V_f - \frac{V}{2}\right) = \frac{1}{4}n_e S v_i (1 - e^{-\frac{V}{2}})
			\end{eqnarray*}
となる. 同様にして$V < 0$のときは$I\left(V_f + \frac{V}{2}\right)$の電流が流れるから, ダブル静電プローブのV-I特性は
			\[
 			 I = \begin{cases}
   					I_{is} (1 - e^{-\frac{V}{2}})  & (V > 0) \\
   					-I_{is} (1 - e^{\frac{V}{2}}) & (V < 0)
 				 \end{cases}
			\]
			$$ただしイオン飽和電流 I_{is} = \frac{1}{4} e n_e S v_i$$
			
と表せるので, グラフは図\ref{tokusei}のようになる. 


\begin{thebibliography}{9}
\bibitem{t} 電気電子情報実験・演習第二テキスト第2分冊 東京大学工学部電気系学科 編
\bibitem{NIST} NIST Atomic Spectra Database Lines Form (https://physics.nist.gov/PhysRefData/ASD/lines\_form.html)
\bibitem{plasma}『プラズマ工学』(1997) 関口忠著 電気学会
\end{thebibliography}


\end{document}
